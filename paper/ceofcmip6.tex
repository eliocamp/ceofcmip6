%Version 2.1 April 2023
% See section 11 of the User Manual for version history
%
%%%%%%%%%%%%%%%%%%%%%%%%%%%%%%%%%%%%%%%%%%%%%%%%%%%%%%%%%%%%%%%%%%%%%%
%%                                                                 %%
%% Please do not use \input{...} to include other tex files.       %%
%% Submit your LaTeX manuscript as one .tex document.              %%
%%                                                                 %%
%% All additional figures and files should be attached             %%
%% separately and not embedded in the \TeX\ document itself.       %%
%%                                                                 %%
%%%%%%%%%%%%%%%%%%%%%%%%%%%%%%%%%%%%%%%%%%%%%%%%%%%%%%%%%%%%%%%%%%%%%

\documentclass[sn-basic,pdflatex]{sn-jnl}

%%%% Standard Packages
%%<additional latex packages if required can be included here>

\usepackage{graphicx}%
\usepackage{multirow}%
\usepackage{amsmath,amssymb,amsfonts}%
\usepackage{amsthm}%
\usepackage{mathrsfs}%
\usepackage[title]{appendix}%
\usepackage{xcolor}%
\usepackage{textcomp}%
\usepackage{manyfoot}%
\usepackage{booktabs}%
\usepackage{algorithm}%
\usepackage{algorithmicx}%
\usepackage{algpseudocode}%
\usepackage{listings}%
%%%%

%%%%%=============================================================================%%%%
%%%%  Remarks: This template is provided to aid authors with the preparation
%%%%  of original research articles intended for submission to journals published
%%%%  by Springer Nature. The guidance has been prepared in partnership with
%%%%  production teams to conform to Springer Nature technical requirements.
%%%%  Editorial and presentation requirements differ among journal portfolios and
%%%%  research disciplines. You may find sections in this template are irrelevant
%%%%  to your work and are empowered to omit any such section if allowed by the
%%%%  journal you intend to submit to. The submission guidelines and policies
%%%%  of the journal take precedence. A detailed User Manual is available in the
%%%%  template package for technical guidance.
%%%%%=============================================================================%%%%

%% Per the spinger doc, new theorem styles can be included using built in style, 
%% but it seems the don't work so commented below
%\theoremstyle{thmstyleone}%
\newtheorem{theorem}{Theorem}%  meant for continuous numbers
%%\newtheorem{theorem}{Theorem}[section]% meant for sectionwise numbers
%% optional argument [theorem] produces theorem numbering sequence instead of independent numbers for Proposition
\newtheorem{proposition}[theorem]{Proposition}%
%%\newtheorem{proposition}{Proposition}% to get separate numbers for theorem and proposition etc.

%% \theoremstyle{thmstyletwo}%
\theoremstyle{remark}
\newtheorem{example}{Example}%
\newtheorem{remark}{Remark}%

%% \theoremstyle{thmstylethree}%
\theoremstyle{definition}
\newtheorem{definition}{Definition}%



\raggedbottom




% tightlist command for lists without linebreak
\providecommand{\tightlist}{%
  \setlength{\itemsep}{0pt}\setlength{\parskip}{0pt}}

% From pandoc table feature
\usepackage{longtable,booktabs,array}
\usepackage{calc} % for calculating minipage widths
% Correct order of tables after \paragraph or \subparagraph
\usepackage{etoolbox}
\makeatletter
\patchcmd\longtable{\par}{\if@noskipsec\mbox{}\fi\par}{}{}
\makeatother
% Allow footnotes in longtable head/foot
\IfFileExists{footnotehyper.sty}{\usepackage{footnotehyper}}{\usepackage{footnote}}
\makesavenoteenv{longtable}




\begin{document}


\title[Article Title runing]{Article Title}

%%=============================================================%%
%% Prefix	-> \pfx{Dr}
%% GivenName	-> \fnm{Joergen W.}
%% Particle	-> \spfx{van der} -> surname prefix
%% FamilyName	-> \sur{Ploeg}
%% Suffix	-> \sfx{IV}
%% NatureName	-> \tanm{Poet Laureate} -> Title after name
%% Degrees	-> \dgr{MSc, PhD}
%% \author*[1,2]{\pfx{Dr} \fnm{Joergen W.} \spfx{van der} \sur{Ploeg} \sfx{IV} \tanm{Poet Laureate}
%%                 \dgr{MSc, PhD}}\email{iauthor@gmail.com}
%%=============================================================%%

\author*[1,2,3]{\fnm{Elio} \sur{Campitelli} }\email{\href{mailto:elio.campitelli@cima.fcen.uba.ar}{\nolinkurl{elio.campitelli@cima.fcen.uba.ar}}}

\author[1,2,3]{\fnm{Leandro B.} \sur{Díaz} }

\author[1,2,3]{\fnm{Carolina} \sur{Vera} }



  \affil*[1]{\orgname{Universidad de Buenos Aires, Facultad de Ciencias Exactas y Naturales, Departamento de Ciencias de la Atmósfera y los Océanos}, \orgaddress{\city{Buenos Aires}, \country{Argentina}}}
  \affil*[2]{\orgname{CONICET -- Universidad de Buenos Aires. Centro de Investigaciones del Mar y la Atmósfera (CIMA)}, \orgaddress{\city{Buenos Aires}, \country{Argentina}}}
  \affil*[3]{\orgname{CNRS -- IRD -- CONICET -- UBA. Instituto Franco-Argentino para el Estudio del Clima y sus Impactos (IRL 3351 IFAECI)}, \orgaddress{\city{Buenos Aires}, \country{Argentina}}}

\abstract{blablabla}

\keywords{key, dictionary, word}


\pacs[JEL Classification]{D8, H51}
\pacs[MSC Classification]{35A01, 65L10}

\maketitle

\hypertarget{intro}{%
\section{Introduction}\label{intro}}

\hypertarget{methods}{%
\section{Data and methods}\label{methods}}

Fuentes de datos.

Describir un poco CMIP6 y DAMIP.

Describir los cEOF brevemente.

Describir A-SAM brevemente

\hypertarget{restuls}{%
\section{Results and discussion}\label{restuls}}

\hypertarget{validation}{%
\subsection{Validation}\label{validation}}

Validación de CMIP6 y de DAMIP.

Figuras de la tesis, inluyendo acá las regresiones con la altura geopotencial.

\hypertarget{relationship-with-tropical-variability}{%
\subsection{Relationship with tropical variability}\label{relationship-with-tropical-variability}}

De la tesis.

Explorar DAMIP. Con suerte no da particularmente distinto y no se muestra.

\hypertarget{relationship-with-sam}{%
\subsection{Relationship with SAM}\label{relationship-with-sam}}

Idem arriba

\hypertarget{trends}{%
\subsection{Trends}\label{trends}}

Tesis.

Calcular las tendencias y su incertidumbre bien, con un modelo mixto.

\hypertarget{conclusions}{%
\subsection{Conclusions}\label{conclusions}}

\bibliography{bibliography.bib}


\end{document}
